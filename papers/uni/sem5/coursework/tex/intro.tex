\chapter*{\large ВВЕДЕНИЕ}  
\addcontentsline{toc}{chapter}{ВВЕДЕНИЕ}
Текущий уровень научно-технического прогресса обеспечивает 
людям свободный
доступ к информации, средствам ее создания и воспроизведения. Исключением не
стала музыка. Повсеместное распространение мобильных устройств и 
Интернета послужило причиной качественных изменений в опыте потребления и
восприятия музыки: слушатель более не ограничен выборкой композиций, предоставляемой
радио или компактными носителями, а волен выбирать из широкого спектра жанров,
представленных на стриминговых сервисах, что позволяет формировать музыкальные
предпочтения более глубоко.

Массовое распространение музыки и повышение ее доступности послужили возникновению 
новых направлений и стилей, а также стали причиной взрывного роста количества композиций, 
доступных в открытых источниках, что может затруднить поиск подходящих композиций.

В связи с данными изменениями становится актуальной задача
классификации музыки и разработка новых рекомендательных
алгоритмов. С упомянутой задачей хорошо справляются нейросетевые
технологии, которые предоставляют широкий спектр инструментов
для решения многочисленных задач, связанных с анализом и обработкой
аудиоданных. Данная работа призвана предоставить обзор
существующих методов классификации музыкальных данных с применением
нейронных сетей и продемонстрировать некоторые из них на
модельных задачах.

Для достижения данной цели были изучены существующие подходы нейросетевого анализа аудиоданных, подготовлен набор данных, состоящий из композиций современных жанров, разработаны два решения, представляющие
различные подходы обучения нейронных сетей, а также произведено сравнение точности использованных алгоритмов.


