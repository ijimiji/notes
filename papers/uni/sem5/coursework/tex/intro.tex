\chapter*{\large ВВЕДЕНИЕ}  
\addcontentsline{toc}{chapter}{ВВЕДЕНИЕ}
Текущий уровень научно-технического прогресса обеспечивает 
людям свободный
доступ к информации и средствам ее создания и воспроизведения. Исключением не
стала музыка. Повсеместное распространение мобильных устройств и 
Интернета послужило причиной качественных изменений в опыте потребления и
восприятия музыки: слушатель более не ограничен выборкой композиций, предоставляемой
радио или компактными носителями, а волен выбирать из широкого спектра жанров,
представленных на стриминговых сервисах, что позволяет формировать музыкальные
предпочтения более глубоко.

В связи с данными изменениями становится актуальной задача
классификации музыки и разработка новых рекомендательных
алгоритмов. С упомянутой задачей хорошо справляются нейросетевые
технологии, которые предоставляют широкий спектр инструментов
для решения многочисленных задач, связанных с анализом и обработкой
аудиоданных. Данная работа призвана предоставить обзор
существующих методов классификации музыкальных данных с применением
нейронных сетей и продемонстрировать некоторые из них на
модельных задачах.


