\chapter*{ \large ЗАКЛЮЧЕНИЕ}
\addcontentsline{toc}{chapter}{ЗАКЛЮЧЕНИЕ}
В ходе данной работы были рассмотрены основные понятия, связанные с нейронными сетями, упомянуты некоторые сферы 
применения нейросетевых технологий, а также приведены распространенные методы обработки аудиоданных для их дальнейшего анализа.

В рамках изучения затронутых понятий применительно к задаче анализа аудиоданных были разработаны два решения для классификации музыки по жанрам, а также составлен набор данных для испытания данных решений. Опытным путем было установлено, что кластеризация без дополнительных эвристик над данными не дает
удовлeтворительных результатов, а использование нейронной сети, обученной с метками жанров, показала более высокую точность.

После изучения других исследований были найдены слабые места построенных моделей и возможные пути их решения, которые лежат в использовании сверточных нейронных сетей
для работы над данными в целом, а не с полученными из характеристиками, что может стать объектом дальнейшего исследования.