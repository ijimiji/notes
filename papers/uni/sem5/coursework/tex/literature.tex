%\chapter*{\large СПИСОК ИСПОЛЬЗОВАННОЙ ЛИТЕРАТУРЫ}
\begin{thebibliography}{99}
\bibitem{vae} Xianxu Hou. Deep Feature Consistent Variational Autoencoder / Xianxu Hou, Linlin Shen, Ke Sun, Guoping Qiu
\bibitem{em} Ветров Д.~П. Курс лекций байесовские методы машинного обучения / Ветров Д.~П., Кропотов Д.~А.
\bibitem{bgu_krasn} Головко, В. А. Нейросетевые технологии обработки данных : учеб. пособие / В. А. Головко, В. В. Краснопрошин. – Минск : БГУ, 2017. – 263 с. – (Классическое университетское издание). – Режим доступа: http://elib.bsu.by/handle/123456789/193558.
\bibitem{sozykin} Созыкин, А. В. Обзор методов обучения глубоких нейронных сетей / А. В. Созыкин // Вестник ЮУрГУ. Серия: Вычислительная математика и информатика. – 2017. – Т. 6, № 3. – С. 28-59. – Режим доступа: https://doi.org/10.14529/cmse170303.
\bibitem{cyber_alex} Алексеев, П.А. Алгоритмы классификации и идентификации аудиозаписей / П.А. Алексеев // Время науки. – 2022. – № 1. – С. 4–10. – Режим доступа: https://cyberleninka.ru/article/n/algoritmy-klassifikatsii-i-identifikatsii-audiozapisey.
\bibitem{bguir_mus} Жвакина, А.В. Нейросетевое распознавание жанра музыкальных произведений / А.В.  Жвакина, В.И. Дектярев // BIG DATA and Advanced Analytics = BIG DATA и анализ высокого уровня : сборник материалов V Международной научно-практической конференции, Минск, 13–14 марта 2019 г. В 2 ч. Ч. 1 / Белорусский государственный университет информатики и радиоэлектроники; редкол. : В.А. Богуш [и др.]. – Минск, 2019. – С. 206–217. – Режим доступа: https://libeldoc.bsuir.by/handle/123456789/34719.
\bibitem{fan} Fan, Q. The Application of Minority Music Style Recognition Based on Deep Convolution Loop Neural Network / Q. Fan // Wireless Communications and Mobile Computing. – 2022. – Vol. 2022. – Article ID 4556135. – 8 p. – Mode of access: https://doi.org/10.1155/2022/4556135.
\bibitem{bguir_rnn} Сычёв, А. Ю. Рекуррентные нейронные сети / Сычёв А. Ю., Стаселько И. Д., Аниховский М. А. // Компьютерные системы и сети : сборник тезисов докладов 56-й научной конференции аспирантов, магистрантов и студентов, Минск, апрель-май 2020 года / Белорусский государственный университет информатики и радиоэлектроники. - Минск : БГУИР, 2020. - С. 163-164.
\bibitem{mcculoch} McCulloch W.S., Pitts W. A Logical Calculus of the Ideas Immanent in Nervous Activity // The Bulletin of Mathematical Biophysics. 1943. Vol. 5, No. 4. P. 115–133.  DOI: 10.1007/BF02478259.
\bibitem{mus_zhao} Nasrullah Z., Zhao Y. Music Artist Classification with Convolutional Recurrent Neural Networks / Z. Nasrullah, Y. Zhao. 2019. - Режим доступа: https://arxiv.org/pdf/1901.04555.pdf.
% TODO
\bibitem{otnes} Отнес Р., Эноксон Л. Прикладной анализ временных рядов. Ч. 1. М., 1982
% TODO
\bibitem{librosa} https://librosa.org/
% TODO
\bibitem{tensorflow} https://github.com/tensorflow/tensorflow
\end{thebibliography}
\addcontentsline{toc}{chapter}{СПИСОК ИСПОЛЬЗОВАННОЙ ЛИТЕРАТУРЫ}
