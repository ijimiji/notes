%\chapter*{\large СПИСОК ИСПОЛЬЗОВАННОЙ ЛИТЕРАТУРЫ}
\begin{thebibliography}{99}
\bibitem{mcculoch} McCulloch W.S., Pitts W. A Logical Calculus of the Ideas Immanent in Nervous Activity // The Bulletin of Mathematical Biophysics. 1943. Vol. 5, No. 4. P. 115–133.  DOI: 10.1007/BF02478259.
\bibitem{sozykin} Созыкин, А. В. Обзор методов обучения глубоких нейронных сетей / А. В. Созыкин // Вестник ЮУрГУ. Серия: Вычислительная математика и информатика. – 2017. – Т. 6, № 3. – С. 28-59. – Режим доступа: https://doi.org/10.14529/cmse170303. -- Дата доступа: 29.11.2022.
\bibitem{cyber_alex} Алексеев, П.А. Алгоритмы классификации и идентификации аудиозаписей / П.А. Алексеев // Время науки. – 2022. – № 1. – С. 4–10. – Режим доступа: https://cyberleninka.ru/article/n/algoritmy-klassifikatsii-i-identifikatsii-audiozapisey.
\bibitem{bguir_rnn} Сычёв, А. Ю. Рекуррентные нейронные сети / Сычёв А. Ю., Стаселько И. Д., Аниховский М. А. // Компьютерные системы и сети : сборник тезисов докладов 56-й научной конференции аспирантов, магистрантов и студентов, Минск, апрель-май 2020 года / Белорусский государственный университет информатики и радиоэлектроники. - Минск : БГУИР, 2020. - С. 163-164.
\bibitem{bgu_krasn} Головко, В. А. Нейросетевые технологии обработки данных : учеб. пособие / В. А. Головко, В. В. Краснопрошин. – Минск : БГУ, 2017. – 263 с. – (Классическое университетское издание). – Режим доступа: http://elib.bsu.by/handle/123456789/193558. -- Дата доступа: 29.11.2022.
\bibitem{vae} Xianxu Hou. Deep Feature Consistent Variational Autoencoder / Xianxu Hou, Linlin Shen, Ke Sun, Guoping Qiu
\bibitem{gtzan} GTZAN [Электронный ресурс] -- Режим доступа: https://www.kaggle.com/datasets/andradaolteanu/gtzan-dataset-music-genre-classification. -- Дата доступа: 29.11.2022.
\bibitem{fft} Быстрое преобразование Фурье // Википедия. [2022]. Режим доступа: https://ru.wikipedia.org/?curid=126120\&oldid=126455198 .  -- Дата доступа: 29.11.2022. 
\bibitem{cyber_zub} Зубаков, А. П. Фурье и вейвлет-преобразования в проблеме распознавания речи // Вестник российских университетов. Математика. 2010. №6. Режим доступа: https://cyberleninka.ru/article/n/furie-i-veyvlet-preobrazovaniya-v-probleme-raspoznavaniya-rechi.  -- Дата доступа: 29.11.2022.
\bibitem{wavelet} Вейвлет-преобразование // Википедия. [2022]. Режим доступа: https://ru.wikipedia.org/?curid=170295\&oldid=119235347 . -- Дата доступа: 29.11.2022.
\bibitem{mus_zhao} Nasrullah Z., Zhao Y. Music Artist Classification with Convolutional Recurrent Neural Networks / Z. Nasrullah, Y. Zhao. 2019. - Режим доступа: https://arxiv.org/pdf/1901.04555.pdf. -- Дата доступа: 29.11.2022.
\bibitem{bguir_mus} Жвакина, А.В. Нейросетевое распознавание жанра музыкальных произведений / А.В.  Жвакина, В.И. Дектярев // BIG DATA and Advanced Analytics = BIG DATA и анализ высокого уровня : сборник материалов V Международной научно-практической конференции, Минск, 13–14 марта 2019 г. В 2 ч. Ч. 1 / Белорусский государственный университет информатики и радиоэлектроники; редкол.: В.А. Богуш [и др.]. – Минск, 2019. – С. 206–217. – Режим доступа: https://libeldoc.bsuir.by/handle/123456789/34719. -- Дата доступа: 29.11.2022.
\bibitem{fan} Fan, Q. The Application of Minority Music Style Recognition Based on Deep Convolution Loop Neural Network / Q. Fan // Wireless Communications and Mobile Computing. – 2022. – Vol. 2022. – Article ID 4556135. – 8 p. – Режим доступа: https://doi.org/10.1155/2022/4556135. -- Дата доступа 29.11.2022.
% TODO
\bibitem{otnes} Прикладной анализ временных рядов: основные методы / Р. Отнес, Л. Эноксон ; перевод с англ. В. И. Хохлова. - Москва : Мир, 1982. - 428 с.
% TODO
\bibitem{librosa} Librosa [Электронный ресурс] -- Режим доступа: https://librosa.org. -- Дата доступа: 29.11.2022.
% TODO
% TODO
\bibitem{tensorflow} Tensorflow [Электронный ресурс] -- Режим доступа: https://github.com/tensorflow/tensorflow. -- Дата доступа: 29.11.2022.
\bibitem{numpy} NumPy [Электронный ресурс] -- Режим доступа: https://numpy.org. -- Дата доступа: 29.11.2022.
\bibitem{lastfm} Last.fm [Электронный ресурс] -- Режим доступа: https://www.last.fm. -- Дата доступа: 29.11.2022.
\bibitem{cyberbred} Бредихин Арсентий Игоревич Алгоритмы обучения сверточных нейронных сетей // Вестник ЮГУ. 2019. №1 (52). Режим доступа: https://cyberleninka.ru/article/n/algoritmy-obucheniya-svertochnyh-neyronnyh-setey. -- Дата доступа: 29.11.2022.
\bibitem{phonk} Фонк // Википедия. [2022]. Режим доступа: https://ru.wikipedia.org/?curid=8372660\&oldid=126173170. -- Дата доступа: 29.11.2022.
\bibitem{wikiconv} Свёрточная нейронная сеть // Википедия. [2022]. -- Режим доступа: https://ru.wikipedia.org/?curid=5075705\&oldid=124511947. -- Дата доступа: 29.11.2022.
\bibitem{adam} Kingma, D.P. Adam: A method for stochastic optimization [Electronical Resource] / D.P. Kingma, J.L. Ba // arXiv preprint. – Режим доступа: https://arxiv.org/abs/1412.6980. -- Дата доступа: 29.11.2022.
\bibitem{fma} Defferrard M., Benzi K., Vandergheynst P., Bresson X. FMA: A Dataset For Music Analysis [Электронный ресурс] // arXiv.org. 2016. -- Режим доступа: https://arxiv.org/abs/1612.01840. -- Дата доступа: 29.11.2022.
\bibitem{kim} Kim, J., Urbano, J., Liem, C.C.S. et al. One deep music representation to rule them all? A comparative analysis of different representation learning strategies. Neural Comput \& Applic 32, 1067–1093 (2020). -- Режим доступа: https://doi.org/10.1007/s00521-019-04076-1. -- Дата доступа: 29.11.2022.
\bibitem{zero} Choi J., Lee J., Park J., Nam J. Zero-shot Learning for Audio-based Music Classification and Tagging [Электронный ресурс] // arXiv.org. 2019. Режим доступа: https://arxiv.org/abs/1907.02670. -- Дата доступа: 29.11.2022.
\end{thebibliography}

\addcontentsline{toc}{chapter}{СПИСОК ИСПОЛЬЗОВАННОЙ ЛИТЕРАТУРЫ}
